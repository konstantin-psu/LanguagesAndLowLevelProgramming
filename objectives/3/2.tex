\documentclass{article}

\usepackage{fancyhdr}
\usepackage{extramarks}
\usepackage{amsmath}
\usepackage{amsthm}
\usepackage{amsfonts}
\usepackage{tikz}
\usepackage[plain]{algorithm}
\usepackage{algpseudocode}
\usepackage{enumerate}

\usepackage{listings}
\usepackage{xcolor}
\usepackage{forest}
\usepackage[shortlabels]{enumitem}
     \setlist[enumerate, 1]{1\textsuperscript{o}}
\lstset { %
    language=C++,
    backgroundcolor=\color{black!5}, % set backgroundcolor
    basicstyle=\footnotesize,% basic font setting
}

%\usetikzlibrary{automata,positioning}
\usetikzlibrary{positioning,shapes,shadows,arrows,automata}

%
% Basic Document Settings
%

\topmargin=-0.45in
\evensidemargin=0in
\oddsidemargin=0in
\textwidth=6.5in
\textheight=9.0in
\headsep=0.25in

\linespread{1.1}

\pagestyle{fancy}
\lhead{\hmwkAuthorName}
\rhead{ (\hmwkClassInstructor\ \hmwkClassTime): \hmwkTitle}
\lfoot{\lastxmark}
\cfoot{\thepage}

\renewcommand\headrulewidth{0.4pt}
\renewcommand\footrulewidth{0.4pt}

\setlength\parindent{0pt}

%
% Create Problem Sections
%

\newcommand{\enterProblemHeader}[1]{
    \nobreak\extramarks{}{Problem \arabic{#1} continued on next page\ldots}\nobreak{}
    \nobreak\extramarks{Problem \arabic{#1} (continued)}{Problem \arabic{#1} continued on next page\ldots}\nobreak{}
}

\newcommand{\exitProblemHeader}[1]{
    \nobreak\extramarks{Problem \arabic{#1} (continued)}{Problem \arabic{#1} continued on next page\ldots}\nobreak{}
    \stepcounter{#1}
    \nobreak\extramarks{Problem \arabic{#1}}{}\nobreak{}
}

\setcounter{secnumdepth}{0}
\newcounter{partCounter}


\newcommand{\hmwkTitle}{Portfolio submission\ \#2, Topic 2}
\newcommand{\hmwkDueDate}{April 15, 2016}
\newcommand{\hmwkClass}{CS510 Languages and Low Level Programming}
\newcommand{\hmwkClassTime}{Spring 2016}
\newcommand{\hmwkClassInstructor}{Mark P. Jones}
\newcommand{\hmwkAuthorName}{Konstantin Macarenco}


\title{
    \vspace{2in}
    \textmd{\textbf{\hmwkClass:\ \hmwkTitle}}\\
    \normalsize\vspace{0.1in}\small{Due\ on\ \hmwkDueDate\ at 11:59pm}\\
    \vspace{0.1in}\large{\textit{\hmwkClassInstructor\ \hmwkClassTime}}
    \vspace{3in}
}

\author{\textbf{\hmwkAuthorName}}
\date{}

\renewcommand{\part}[1]{\textbf{\large Part \Alph{partCounter}}\stepcounter{partCounter}\\}

%
% Various Helper Commands
%

% Useful for algorithms
\newcommand{\alg}[1]{\textsc{\bfseries \footnotesize #1}}

% For derivatives
\newcommand{\deriv}[1]{\frac{\mathrm{d}}{\mathrm{d}x} (#1)}

% For partial derivatives
\newcommand{\pderiv}[2]{\frac{\partial}{\partial #1} (#2)}

% Integral dx
\newcommand{\dx}{\mathrm{d}x}

% Alias for the Solution section header
\newcommand{\solution}{\textbf{\large Solution}}

% Probability commands: Expectation, Variance, Covariance, Bias
\newcommand{\E}{\mathrm{E}}
\newcommand{\Var}{\mathrm{Var}}
\newcommand{\Cov}{\mathrm{Cov}}
\newcommand{\Bias}{\mathrm{Bias}}

\begin{document}

\maketitle

\pagebreak

%\begin{enumerate}[(a), leftmargin = 0.7cm, nosep]
\textbf{Attempted Topic: }\\
Discuss common challenges in low-level systems software development, including debugging in
a bare metal environment.\\

Narrative
\vspace{0.5cm}

%CCCCOOOO

There are many areas where low-level systems software development is used, for many
different reasons, like OS performance optimization, removing middle layer software due to strict
hardware requirements (like embedded systems), necessity of a small footprint etc. \\

\indent
Programming in low level mode proves to be difficult and error prone. It requires high
attention to small details and good knowledge of the used hardware. There is very little, if 
none,
abstraction that most of the programmers normally rely on, most of the OS facilities and API 
are not
available. Additional challenge is the lack of trivial debugging tools (not talking about
debugger), sometimes simple methods, like use of printf(), are out of consideration. 
For example a
common debugging technique in low level applications for Raspberry PI, is to blink attached
LED light, which is similar to POST beep codes, which BTW according to specs are not always
guaranteed.  There is no easy way to go through program's logic, inspect memory and variables 
values.
Deep understanding of the used hardware, such as ABI, CPU
Registers layout, Boot Process, Memory Management, Context Switching, Interrupts etc. is 
vital.  
Learning these technical details is a dull and meticulous task, for example Intel x86-32 
documentation is more than 3000 pages long.
\\

Some Low level systems debugging approaches.
\begin{enumerate}[1.]
\item Run low bare metal application within instrumented virtual machine that emulate desired 
computer architecture, and provide similar functionality as a debugger.
    There are some issues with this approach: this type of ``Debuggers'' are
usually costly (up to \$10+k).
\item Another approach is to create code that will
work equally under bare metal systems and under normal OS, with only recompilation required
to transfer from one to another, however this is not always achievable, since low level 
environments are dramatically more sensitive to undefined
behavior, especially uninitialized variables. This problem can be somewhat mitigated by 
using all
possible levels of compiler optimization, and checks. After all, the products will still 
require thorough testing on the actual hardware.
\item QEMU + gdb. QEMU provides integration with gdb, it acts as a remote gdb server. 
Kernel needs to be compiled with debug symbols (``-g'' flag). One downside is that
debugging symbols will significantly increase kernel's size.
\end{enumerate}

Modern high level languages achieve better productivity and quality, over low level
languages like C and assembly.  They considered to be ``safe'' with much better memory
management, garbage collection, and other feature that are not available for Low-
Level systems programming. 
C and assembly are very powerful,
but unforgiving to inexperienced programmers, they require lots of management and easy
to make mistakes. The situation is exacerbated by inability to use Standard C-Library,
since it heavily relies on OS features. In bare metal means that you start with very
little external interfaces, no console, no file system etc.
This problem is especially significant in environments without common
debugging tools, like bare metal systems.

%CCCCOOO1

\end{document}
