\documentclass{article}

\usepackage{fancyhdr}
\usepackage{extramarks}
\usepackage{amsmath}
\usepackage{amsthm}
\usepackage{amsfonts}
\usepackage{tikz}
\usepackage[plain]{algorithm}
\usepackage{algpseudocode}
\usepackage{enumerate}

\usepackage{listings}
\usepackage{xcolor}
\usepackage{forest}
\usepackage[shortlabels]{enumitem}
     \setlist[enumerate, 1]{1\textsuperscript{o}}
\lstset { %
    language=C++,
    backgroundcolor=\color{black!5}, % set backgroundcolor
    basicstyle=\footnotesize,% basic font setting
}

%\usetikzlibrary{automata,positioning}
\usetikzlibrary{positioning,shapes,shadows,arrows,automata}

%
% Basic Document Settings
%

\topmargin=-0.45in
\evensidemargin=0in
\oddsidemargin=0in
\textwidth=6.5in
\textheight=9.0in
\headsep=0.25in

\linespread{1.1}

\pagestyle{fancy}
\lhead{\hmwkAuthorName}
\rhead{ (\hmwkClassInstructor\ \hmwkClassTime)}
\lfoot{\lastxmark}
\cfoot{\thepage}

\renewcommand\headrulewidth{0.4pt}
\renewcommand\footrulewidth{0.4pt}

\setlength\parindent{0pt}

%
% Create Problem Sections
%

\newcommand{\enterProblemHeader}[1]{
    \nobreak\extramarks{}{Problem \arabic{#1} continued on next page\ldots}\nobreak{}
    \nobreak\extramarks{Problem \arabic{#1} (continued)}{Problem \arabic{#1} continued on next page\ldots}\nobreak{}
}

\newcommand{\exitProblemHeader}[1]{
    \nobreak\extramarks{Problem \arabic{#1} (continued)}{Problem \arabic{#1} continued on next page\ldots}\nobreak{}
    \stepcounter{#1}
    \nobreak\extramarks{Problem \arabic{#1}}{}\nobreak{}
}

\setcounter{secnumdepth}{0}
\newcounter{partCounter}


\newcommand{\hmwkTitle}{Portfolio submission Topic 3: \\
        Explain how conventional operating system features (multiple address spaces, context 
        switching, protection, etc.) motivate the desire for (and benefit from) hardware 
        support. }
\newcommand{\hmwkDueDate}{April 29, 2016}
\newcommand{\hmwkClass}{CS510 Languages and Low Level Programming}
\newcommand{\hmwkClassTime}{Spring 2016}
\newcommand{\hmwkClassInstructor}{Mark P. Jones}
\newcommand{\hmwkAuthorName}{Konstantin Macarenco}


\title{
    \vspace{2in}
    \textmd{\textbf{\hmwkClass: \hmwkTitle}}\\
    \normalsize\vspace{0.1in}\small{Due\ on\ \hmwkDueDate\ at 11:59pm}\\
    \vspace{0.1in}\large{\textit{\hmwkClassInstructor\ \hmwkClassTime}}
    \vspace{3in}
}

\author{\textbf{\hmwkAuthorName}}
\date{}

\renewcommand{\part}[1]{\textbf{\large Part \Alph{partCounter}}\stepcounter{partCounter}\\}

%
% Various Helper Commands
%

% Useful for algorithms
\newcommand{\alg}[1]{\textsc{\bfseries \footnotesize #1}}

% For derivatives
\newcommand{\deriv}[1]{\frac{\mathrm{d}}{\mathrm{d}x} (#1)}

% For partial derivatives
\newcommand{\pderiv}[2]{\frac{\partial}{\partial #1} (#2)}

% Integral dx
\newcommand{\dx}{\mathrm{d}x}

% Alias for the Solution section header
\newcommand{\solution}{\textbf{\large Solution}}

% Probability commands: Expectation, Variance, Covariance, Bias
\newcommand{\E}{\mathrm{E}}
\newcommand{\Var}{\mathrm{Var}}
\newcommand{\Cov}{\mathrm{Cov}}
\newcommand{\Bias}{\mathrm{Bias}}

\begin{document}

\maketitle

\pagebreak


        \vspace{1cm}

%CCSS

        The main benefit of implementing some of the OS features in hardware is
        performance and security.
        Specialized hardware will be in most of the cases many times faster than general
        software. Some tasks are simply not feasible without some level or hardware
        support. 

        Software approach reduces cost and complexity of hardware centric systems. And
        vice versa performance of software is increased by special purpose hardware.

        When some concept or a feature becomes widely adopted there is a good chance that it
        will be adopted in hardware, however it is not always the case, or at least not
        always successful (context switching).


        \textbf{Memory Management}\\

        Programming first computers without hardware's aid was a big challenge.
        What to do when a task runs out of memory? How to fit a large applications into
        available RAM? How to protect one task from corrupting another's task memory? Etc.

        Memory Management hardware infrastructure development was driven by the 
        concept of Virtual Memory which required to support of multiple address spaces, and 
        need of fast dynamic translation between virtual and physical addresses. 
        This is a hard and complex task that would be too slow and prone to security problems
        if implemented in pure software. 

        So specific hardware was created to solve Virtual Memory problems, which include but
        limited to
        \begin{itemize}
            \item Fast memory translation.
            \item Address Space Switch.
            \item Address Space isolation.
            \item Memory overflow protection (swapping).
        \end{itemize}

        Memory space isolation was a big problem in earlier OSs.
        Memory protection tries to solve problem of a process affecting other processes or the OS
        memory. Without hardware support this problem is impossible to solve since during  execution
        the current process is in total control and can do anything without proper protection
        mode. Protection rings, segmentation and paging enable this feature.
        OS creates page table and changes cr3 to point to the current page table. This is
        a privileged instruction and can be only done in ring0. \\

        This level of protection is impossible without proper hardware support, I see two possible 
        software solutions (neither is as good as having hardware support):
        \begin{itemize}
            \item An OS provides virtualized environment (virtual machine that implements
                hardware protection). This approach is close to be as good as real hardware
                protection but struggles from performance problems.
            \item The OS can periodically scan memory for violations. This approach is more of
                a debugging tool, it is a lot less secure and doesn't prevent from interprocess 
                memory access.
        \end{itemize}

        \textbf{Context Switching}\\
        
        I was surprised to learn, while researching this topic,
        that context switch in modern OSs is a done by software.
        Intel task switch mechanism nowadays is used mainly to transfer from one
        protection level to another. Linux creates only one kernel TSS since it's required
        to enable protection  and uses only 
        ESP0 and SS0 to switch between security rings. 
        Reasons behind using software - hardware context switch is appeared to be slower
        less flexible and not portable from x86 to x86\_64. X86 during context switch
        saves lots of information that not all applications need, making already costly
        operation even worse.\\

        \textbf{Conclusion}

        To be adopted in hardware a concept must be widely accepted an versatile enough not
        to introduce any unnecessary assumptions. Memory management is a good example of
        such a feature, and context switching, is the opposite.


%CCEE
        % When feature becomes permanent, or in high demand, such as context switching 

\end{document}
