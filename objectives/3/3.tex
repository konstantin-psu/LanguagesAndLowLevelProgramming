\documentclass{article}

\usepackage{fancyhdr}
\usepackage{extramarks}
\usepackage{amsmath}
\usepackage{amsthm}
\usepackage{amsfonts}
\usepackage{tikz}
\usepackage[plain]{algorithm}
\usepackage{algpseudocode}
\usepackage{enumerate}

\usepackage{listings}
\usepackage{xcolor}
\usepackage{forest}
\usepackage[shortlabels]{enumitem}
     \setlist[enumerate, 1]{1\textsuperscript{o}}
\lstset { %
    language=C++,
    backgroundcolor=\color{black!5}, % set backgroundcolor
    basicstyle=\footnotesize,% basic font setting
}

%\usetikzlibrary{automata,positioning}
\usetikzlibrary{positioning,shapes,shadows,arrows,automata}

%
% Basic Document Settings
%

\topmargin=-0.45in
\evensidemargin=0in
\oddsidemargin=0in
\textwidth=6.5in
\textheight=9.0in
\headsep=0.25in

\linespread{1.1}

\pagestyle{fancy}
\lhead{\hmwkAuthorName}
\rhead{ (\hmwkClassInstructor\ \hmwkClassTime)}
\lfoot{\lastxmark}
\cfoot{\thepage}

\renewcommand\headrulewidth{0.4pt}
\renewcommand\footrulewidth{0.4pt}

\setlength\parindent{0pt}

%
% Create Problem Sections
%

\newcommand{\enterProblemHeader}[1]{
    \nobreak\extramarks{}{Problem \arabic{#1} continued on next page\ldots}\nobreak{}
    \nobreak\extramarks{Problem \arabic{#1} (continued)}{Problem \arabic{#1} continued on next page\ldots}\nobreak{}
}

\newcommand{\exitProblemHeader}[1]{
    \nobreak\extramarks{Problem \arabic{#1} (continued)}{Problem \arabic{#1} continued on next page\ldots}\nobreak{}
    \stepcounter{#1}
    \nobreak\extramarks{Problem \arabic{#1}}{}\nobreak{}
}

\setcounter{secnumdepth}{0}
\newcounter{partCounter}


\newcommand{\hmwkTitle}{Portfolio submission Topic 3: \\
        Explain how conventional operating system features (multiple address spaces, context 
        switching, protection, etc.) motivate the desire for (and benefit from) hardware 
        support. }
\newcommand{\hmwkDueDate}{April 29, 2016}
\newcommand{\hmwkClass}{CS510 Languages and Low Level Programming}
\newcommand{\hmwkClassTime}{Spring 2016}
\newcommand{\hmwkClassInstructor}{Mark P. Jones}
\newcommand{\hmwkAuthorName}{Konstantin Macarenco}


\title{
    \vspace{2in}
    \textmd{\textbf{\hmwkClass:\ hmwkTitle}}\\
    \normalsize\vspace{0.1in}\small{Due\ on\ \hmwkDueDate\ at 11:59pm}\\
    \vspace{0.1in}\large{\textit{\hmwkClassInstructor\ \hmwkClassTime}}
    \vspace{3in}
}

\author{\textbf{\hmwkAuthorName}}
\date{}

\renewcommand{\part}[1]{\textbf{\large Part \Alph{partCounter}}\stepcounter{partCounter}\\}

%
% Various Helper Commands
%

% Useful for algorithms
\newcommand{\alg}[1]{\textsc{\bfseries \footnotesize #1}}

% For derivatives
\newcommand{\deriv}[1]{\frac{\mathrm{d}}{\mathrm{d}x} (#1)}

% For partial derivatives
\newcommand{\pderiv}[2]{\frac{\partial}{\partial #1} (#2)}

% Integral dx
\newcommand{\dx}{\mathrm{d}x}

% Alias for the Solution section header
\newcommand{\solution}{\textbf{\large Solution}}

% Probability commands: Expectation, Variance, Covariance, Bias
\newcommand{\E}{\mathrm{E}}
\newcommand{\Var}{\mathrm{Var}}
\newcommand{\Cov}{\mathrm{Cov}}
\newcommand{\Bias}{\mathrm{Bias}}

\begin{document}

\maketitle

\pagebreak


        \vspace{1cm}

%CCSS

        The main benefit of implementing some of the OS features in hardware is
        performance and security.
        Specialized hardware will be in most of the cases many times faster than general
        software. Some tasks are simply not feasible without some level or hardware
        support. 

        Software approach reduces cost and complexity of hardware centric systems. And
        vice versa performance of software is increased by special purpose hardware.\\

        \textbf{Memory Management}\\

        Programming first computers without any memory management aid was a big headache.
        What to do when a task runs out of memory? How to feet larger applications into
        available RAM? How to protect one task from corrupting another's task memory? Etc.\\

        Memory Management Infrastructure development was driven by the 
        concept of Virtual Memory which required to support of multiple address spaces.
        Dynamic translation between virtual and physical addresses. This concept would be
        too slow and prone to security problems if implemented in pure software. 
        Memory management hardware solves following issues related to Virtual Memory and
        Address spaces:
        \begin{itemize}
            \item Fast memory translation.
            \item Address Space Switch.
            \item Address Space isolation.
            \item Memory overflow protection (swapping).
        \end{itemize}

        Memory protection tries to solve problem of a process affecting other processes or the OS
        memory. Without hardware support this problem is impossible to solve since during  execution
        cap current process is in total control and can do anything without proper protection
        mode. Protection rings, segmentation and paging enable this feature.
        OS creates page table and changes cr3 to point to the current page table. This is
        a privileged instruction and can be only done in ring0. \\

        I see two possible solutions without hardware support (neither is as good as
        having hardware support)
        \begin{itemize}
            \item An OS provides virtualised environment (virtual machine that implements
                hardware protection). This approach raises some performance problems.
            \item OS can periodically scan memory for violation. This approach is more of
                a debugging tool, it is a lot less secure and doesn't prevent from interprocess 
                memory access.
        \end{itemize}

        \textbf{Context Switching}\\
        
        I was surprised to learn that context switch in modern OSs is a software
        feature. Intel task switch mechanism nowadays is used mainly to transfer from one
        protection level to another. Linux creates only on TSS(segmentation related
        feature) for this purpose and uses
        ESP0 and SS0 for this purpose. Hardware context switch is appeared to be slower
        less flexible and portable from x86 to x86\_64.\\



%CCEE
        % When feature becomes permanent, or in high demand, such as context switching 

\end{document}
