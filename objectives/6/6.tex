\documentclass{article}

\usepackage{fancyhdr}
\usepackage{extramarks}
\usepackage{amsmath}
\usepackage{amsthm}
\usepackage{amsfonts}
\usepackage{tikz}
\usepackage[plain]{algorithm}
\usepackage{algpseudocode}
\usepackage{enumerate}

\usepackage{listings}
\usepackage{xcolor}
\usepackage{forest}
\usepackage[shortlabels]{enumitem}
\setlist[enumerate, 1]{1\textsuperscript{o}}
\lstset { %
    language=C++,
        backgroundcolor=\color{black!5}, % set backgroundcolor
            basicstyle=\footnotesize,% basic font setting
}

%\usetikzlibrary{automata,positioning}
\usetikzlibrary{positioning,shapes,shadows,arrows,automata}

%
% Basic Document Settings
%

\topmargin=-0.45in
\evensidemargin=0in
\oddsidemargin=0in
\textwidth=6.5in
\textheight=9.0in
\headsep=0.25in

\linespread{1.1}

\pagestyle{fancy}
\lhead{\hmwkAuthorName}
\rhead{ (\hmwkClassInstructor\ \hmwkClassTime): \hmwkTitle}
\lfoot{\lastxmark}
\cfoot{\thepage}

\renewcommand\headrulewidth{0.4pt}
\renewcommand\footrulewidth{0.4pt}

\setlength\parindent{0pt}

\setcounter{secnumdepth}{0}
\newcounter{partCounter}


\newcommand{\hmwkTitle}{Portfolio submission, Topic 6}
\newcommand{\hmwkDueDate}{May 13, 2016}
\newcommand{\hmwkClass}{CS510 Languages and Low Level Programming}
\newcommand{\hmwkClassTime}{Spring 2016}
\newcommand{\hmwkClassInstructor}{Mark P. Jones}
\newcommand{\hmwkAuthorName}{Konstantin Macarenco}


\title{
    \vspace{2in}
    \textmd{\textbf{\hmwkClass:\ \hmwkTitle}}\\
        \normalsize\vspace{0.1in}\small{Due\ on\ \hmwkDueDate\ at 11:59pm}\\
        \vspace{0.1in}\large{\textit{\hmwkClassInstructor\ \hmwkClassTime}}
    \vspace{3in}
}

\author{\textbf{\hmwkAuthorName}}
\date{}

\renewcommand{\part}[1]{\textbf{\large Part \Alph{partCounter}}\stepcounter{partCounter}\\}

\begin{document}

\maketitle

\pagebreak

%\begin{enumerate}[(a), leftmargin = 0.7cm, nosep]
\section{Topic 6.}  Describe the motivation, implementation, and application of microkernel
abstractions for managing address spaces, threads, and interprocess communication (IPC). \\

             \subsection{Motivation}
             %STARTSTART
             Idea behind microkernel is to provide ``policy free'' environment, with no assumptions about
             services it will be running. All functionality of a traditional kernel is moved out, and
             implemented as a set of independent services. Microkernel provides only minimum set of
             abstraction over hardware, such as
             \begin{enumerate}[-, leftmargin = 0.7cm, nosep]
    \item Address Spacing
\item Inter-process Communication (IPC)
    \item Thread Management
    \item Unique Identifiers
    \end{enumerate}
    Systems that use this approach will implement only the functionality needs to operate, have
    smaller footprint. All nonessential for the kernel functions, like network support, are
    implemented as services. Each service has it's own address space, communication with other
    services allowed only through Only kernel functionality is in privileged mode (ring 0). 
    \vspace{0.2cm}

    Advantages
    \begin{enumerate}[-, leftmargin = 0.7cm, nosep]
    \item Robustness, a service crash doesn't affect any other part of the system.
    \item Easier maintenance.
    \item Enforces modular structure.
    \item Small size of Trusted Computing Base.
    \end{enumerate}
    \vspace{0.2cm}

    Disadvantages
    \begin{enumerate}[-, leftmargin = 0.7cm, nosep]
    \item Performance loss due to high overhead of services interaction.
    \item Larger memory footprint.
    \item Complicated process management.
    \end{enumerate}

    \subsection{Implementation}
    \textbf{History}\\

    Mach\\

    L3\\

    L4\\

    \textbf{Implementation details}
    \begin{enumerate}[-, leftmargin = 0.7cm, nosep]
    \item Userspace perspective \\
        Kernel Information Page (kernel version, host system info, address space layout
        info, system call entry points)
        
        Predefined KIP adddress

        Slow system call LOCK NOP -> illegal opcode exception -> kernel -> load KIP address
        to context registers -> return to user mode
        \\ 
        and User Thread COntorl Block (One UTCB for eac thread) in the address space
         Message registers (MRs), and thread control registers (TCRs) 

         all UTCBs are in UTCB area

         64 message registers $MR_0 - MR_63$

         Miscellaneous fields: error code, exceptin handler pager acceptor
         
         UTCB address  points to the middle of the UTCB \\

         User processes can read/write data to any UTCB available

         thread priority is stored in protected structure, any data read form UTCB cannot be
         trusted and must  be validated by the kernel before use.

         UTCB mappings created by kernel



    \end{enumerate}

    %ENDEND


    \subsection{Application}
    \end{document}
