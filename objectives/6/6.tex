\documentclass{article}

\usepackage{fancyhdr}
\usepackage{extramarks}
\usepackage{amsmath}
\usepackage{amsthm}
\usepackage{amsfonts}
\usepackage{tikz}
\usepackage[plain]{algorithm}
\usepackage{algpseudocode}
\usepackage{enumerate}

\usepackage{listings}
\usepackage{xcolor}
\usepackage{forest}
\usepackage[shortlabels]{enumitem}
\setlist[enumerate, 1]{1\textsuperscript{o}}
\lstset { %
    language=C++,
        backgroundcolor=\color{black!5}, % set backgroundcolor
            basicstyle=\footnotesize,% basic font setting
}

%\usetikzlibrary{automata,positioning}
\usetikzlibrary{positioning,shapes,shadows,arrows,automata}

%
% Basic Document Settings
%

\topmargin=-0.45in
\evensidemargin=0in
\oddsidemargin=0in
\textwidth=6.5in
\textheight=9.0in
\headsep=0.25in

\linespread{1.1}

\pagestyle{fancy}
\lhead{\hmwkAuthorName}
\rhead{ (\hmwkClassInstructor\ \hmwkClassTime): \hmwkTitle}
\lfoot{\lastxmark}
\cfoot{\thepage}

\renewcommand\headrulewidth{0.4pt}
\renewcommand\footrulewidth{0.4pt}

\setlength\parindent{0pt}

\setcounter{secnumdepth}{0}
\newcounter{partCounter}


\newcommand{\hmwkTitle}{Portfolio submission, Topic 6}
\newcommand{\hmwkDueDate}{May 13, 2016}
\newcommand{\hmwkClass}{CS510 Languages and Low Level Programming}
\newcommand{\hmwkClassTime}{Spring 2016}
\newcommand{\hmwkClassInstructor}{Mark P. Jones}
\newcommand{\hmwkAuthorName}{Konstantin Macarenco}


\title{
    \vspace{2in}
    \textmd{\textbf{\hmwkClass:\ \hmwkTitle}}\\
        \normalsize\vspace{0.1in}\small{Due\ on\ \hmwkDueDate\ at 11:59pm}\\
        \vspace{0.1in}\large{\textit{\hmwkClassInstructor\ \hmwkClassTime}}
    \vspace{3in}
}

\author{\textbf{\hmwkAuthorName}}
\date{}

\renewcommand{\part}[1]{\textbf{\large Part \Alph{partCounter}}\stepcounter{partCounter}\\}

\begin{document}

\maketitle

\pagebreak

%\begin{enumerate}[(a), leftmargin = 0.7cm, nosep]
        \section{Topic 6.}  Describe the motivation, implementation, and application of microkernel
        abstractions for managing address spaces, threads, and interprocess communication (IPC). \\

             \subsection{Motivation}
             %STARTSTART
             Idea behind microkernel is to provide ``policy free'' environment, with no assumptions about
             services it will be running. All functionality of a traditional kernel is moved out, and
             implemented as a set of independent services. Microkernel provides only minimum set of
             abstraction over hardware, such as
             \begin{enumerate}[-, leftmargin = 0.7cm, nosep]
                 \item Address Spacing
                 \item Inter-process Communication (IPC)
                 \item Thread Management
                 \item Unique Identifiers
             \end{enumerate}
        Systems that use this approach will implement only the functionality needs to operate, have
        smaller footprint. All nonessential for the kernel functions, like network support, are
        implemented as services. Each service has it's own address space, communication with other
        services allowed only through Only kernel functionality is in privileged mode (ring 0). 
        \vspace{0.2cm}

        Advantages
        \begin{enumerate}[-, leftmargin = 0.7cm, nosep]
            \item Robustness, a service crash doesn't affect any other part of the system.
            \item Easier maintenance.
            \item Enforces modular structure.
            \item Small size of Trusted Computing Base.
        \end{enumerate}
        \vspace{0.2cm}

        Disadvantages
        \begin{enumerate}[-, leftmargin = 0.7cm, nosep]
            \item Performance loss due to high overhead of services interaction.
        \item Larger memory footprint.  \item Complicated process management.  \end{enumerate}

            Earlier versions of microkernel implementation suffered from high IPC overhead and
        did not look promising. L3/L4 (Jochen Liedtke) implementation proved that microkernel can be successful.
        Liedtke IPC design was 20 times faster than Mach's (fist generation of microkernel).


        \pagebreak

        \subsection{Implementation details}
        L4 implemented as a single linux server, and multiplexed tasks for user processesm
        which means that each task is also a server for it's child tasks, basically acting as Virtual CPU.

        On booting Linux server request memory from it's pager which maps physical memory in
        the server's address space from it's pager which maps physical memory in the
        server's address space, then acts as the pager for the user processes it creates.

        Each task consists of Threads, memory objects and address space. 
        
        \begin{enumerate}[1., leftmargin = 0.7cm, nosep]
            \item \textbf{Address Spacing} \\
                Microkernel implement recursive address space model.

                Initial address space $\sigma_0$ represents all physical memory available. Each
                Threads
                basic memory operations are granting mapping and unmapping, which can be done only
                on the address space mapped (available to the process), e.g. Root process can map
                entire physical memory.

                Hardware interrupts are handled by user level processes, and triggered by the L4
                kernel IPC.
                Each process is a server for it's children, and is responsible for handling page
                faults of the processes it created. When pagefalut occurs, typically thread's
                pager creates new mapping and send it to faulty thread.

                Microkernel supports flexpages - simple pages that can be varied in size. In
                practice all pages larger than 4KB are just a collection of multiple 4K
                pages.

            \item \textbf{Thread Management}\\

                L4 uses thread multiplexing mechanism, with thread0 being the first thread. Each
                thread can create new thread's and multiplex it's resources.

                Each thread has associated Pager, Exception Handler and Scheduler threads.
                For handling page faults, exceptions and scheduling. Schedule threads can set
                scheduling parameters, like priority, time slice and quantum. Time slices can
                be donated to any other thread, or to kernel through donating time to
                nilthreads.

                Kernel Information Page (kernel version, host system info, address space layout
                info, system call entry points), is available to all threads through either a
                predefined address or, through LOCK NOP call.


                Each User has User Thread Control Block (UTCB), which is used for communication
                with kernel.

                Threads can be referenced by thread id's. System defines global Thread ID,
                global interrupt ID, local thread ID, millthread, anythread, and
                anylocalthread. Global Ids are considered to be a bad practice since any thread
                can reference any other thread by it's local id, this potentially can lead to
                DOS attacks. Even if global thread ID is unknown it can be guessed (this is one
                of the problems seL4 tries to solve by using capabilities).

                Thread management is done through ThreadControl system calls.

            \item \textbf{Inter-process Communication (IPC)}
                System calls are expensive, therefore should be avoided. IPC is based on this
                principle and combines send and receive into a single system call.
                Before IPC system call thread loads message registers into the UTCB with a
                message to send, after IPC is complete sender thread is resumed, and it can
                read response from the receiver by reading UTCB registers.

                IPC can be blocking or unblocking depending on implementation. Some message
                registers are passed in CPU, if message is small enough to fit into real
                registers IPC call performance will be a lot faster. Kernel implementation uses
                this since most of the messages \%60 to \%70 percent are small. Larger messages
                transfer is a lot slower. In earlier microkernel generations it was done by
                double copying message from sender to kernel and to receiver, which requires
                TLB flush. L4 avoids this by
                sending a mapping from target's area to sender, temporarily sharing region of
                memory.

        \end{enumerate}

        %ENDEND

    \pagebreak

    \subsection{Application}

    Best application for Microkernel based OSs is Embedded and VM system.
\end{document}
