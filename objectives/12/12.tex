\documentclass{article}

\usepackage{fancyhdr}
\usepackage{extramarks}
\usepackage{amsmath}
\usepackage{amsthm}
\usepackage{amsfonts}
\usepackage{tikz}
\usepackage[plain]{algorithm}
\usepackage{algpseudocode}
\usepackage{enumerate}

\usepackage{listings}
\usepackage{xcolor}
\usepackage{forest}
\usepackage[shortlabels]{enumitem}
\setlist[enumerate, 1]{1\textsuperscript{o}}
\lstset { %
    language=C++,
        backgroundcolor=\color{black!5}, % set backgroundcolor
            basicstyle=\footnotesize,% basic font setting
}

%\usetikzlibrary{automata,positioning}
\usetikzlibrary{positioning,shapes,shadows,arrows,automata}

%
% Basic Document Settings
%

\topmargin=-0.45in
\evensidemargin=0in
\oddsidemargin=0in
\textwidth=6.5in
\textheight=9.0in
\headsep=0.25in

\linespread{1.1}

\pagestyle{fancy}
\lhead{\hmwkAuthorName}
\rhead{ (\hmwkClassInstructor\ \hmwkClassTime): \hmwkTitle}
\lfoot{\lastxmark}
\cfoot{\thepage}

\renewcommand\headrulewidth{0.4pt}
\renewcommand\footrulewidth{0.4pt}

\setlength\parindent{0pt}

\setcounter{secnumdepth}{0}
\newcounter{partCounter}


\newcommand{\hmwkTitle}{Portfolio submission, Topic 12}
\newcommand{\hmwkDueDate}{June 3, 2016}
\newcommand{\hmwkClass}{CS510 Languages and Low Level Programming}
\newcommand{\hmwkClassTime}{Spring 2016}
\newcommand{\hmwkClassInstructor}{Mark P. Jones}
\newcommand{\hmwkAuthorName}{Konstantin Macarenco}


\title{
    \vspace{2in}
    \textmd{\textbf{\hmwkClass:\ \hmwkTitle}}\\
        \normalsize\vspace{0.1in}\small{Due\ on\ \hmwkDueDate\ at 11:59pm}\\
        \vspace{0.1in}\large{\textit{\hmwkClassInstructor\ \hmwkClassTime}}
    \vspace{3in}
}

\author{\textbf{\hmwkAuthorName}}
\date{}

\renewcommand{\part}[1]{\textbf{\large Part \Alph{partCounter}}\stepcounter{partCounter}\\}

\begin{document}

\maketitle

\pagebreak

%\begin{enumerate}[(a), leftmargin = 0.7cm, nosep]
        \section{Topic 12. Explain how the requirements of low-level systems programming motivate the desire
        for (and benefit from) language based support.}

%STARTSTART
        Low-Level system programming is difficult and error prone, developing software in this
        environment is tedious. Languages used for LLP, like Assembly and C  (especially
        Assembly) are very powerful but do very little when it comes to error checking. A
        special domain specific language/compiler would greatly simplify system programming.
        Typical programming errors like buffer overflow, division by zero, null pointer
        dereference etc. are laborious to detect without proper operating system support.
        LLP specific language should be able to detect them during compilation, rather than
        runtime.
        \\

        LLP has many repetitive, though hardware specific tasks like setting Page Directory,
        and Interrupt Vector Table, etc which are purely mechanical by nature and need lots of
        bits manipulations. For historical reasons bit patterns in various parts of the system
        are very different and unpredictable due to backward compatibility requirement. Many
        of these operations would be a lot easier, faster and reliable if supported by the
        language. 
        Another challenge in Low Level Programming is re usability of code - it is hard to
        achieve since every application is restricted  to specific hardware, there is no
        Kernel API that general programmers are used to. 
        However many of the problems can be abstracted by the Language or Standard Library.

        Understanding how different parts of the LLP application are tied together can be
        problematic, since vital parts of code spread out among multiple domain specific
        languages and physical locations, also functionality interactions can happen at
        unexpected places or be hidden, for example by interrupts, or system calls.

        \vspace{0.5cm}
        Would be interesting to talk to someone from big hardware manufacturing company to see
        if there is any proprietary tools exists for this purpose.
       

%ENDEND
\end{document}
