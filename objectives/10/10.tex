\documentclass{article}

\usepackage{fancyhdr}
\usepackage{extramarks}
\usepackage{amsmath}
\usepackage{amsthm}
\usepackage{amsfonts}
\usepackage{tikz}
\usepackage[plain]{algorithm}
\usepackage{algpseudocode}
\usepackage{enumerate}

\usepackage{listings}
\usepackage{xcolor}
\usepackage{forest}
\usepackage[shortlabels]{enumitem}
\setlist[enumerate, 1]{1\textsuperscript{o}}
\lstset { %
    language=C++,
        backgroundcolor=\color{black!5}, % set backgroundcolor
            basicstyle=\footnotesize,% basic font setting
}

%\usetikzlibrary{automata,positioning}
\usetikzlibrary{positioning,shapes,shadows,arrows,automata}

%
% Basic Document Settings
%

\topmargin=-0.45in
\evensidemargin=0in
\oddsidemargin=0in
\textwidth=6.5in
\textheight=9.0in
\headsep=0.25in

\linespread{1.1}

\pagestyle{fancy}
\lhead{\hmwkAuthorName}
\rhead{ (\hmwkClassInstructor\ \hmwkClassTime): \hmwkTitle}
\lfoot{\lastxmark}
\cfoot{\thepage}

\renewcommand\headrulewidth{0.4pt}
\renewcommand\footrulewidth{0.4pt}

\setlength\parindent{0pt}

\setcounter{secnumdepth}{0}
\newcounter{partCounter}


\newcommand{\hmwkTitle}{Portfolio submission, Topic 10}
\newcommand{\hmwkDueDate}{June 3, 2016}
\newcommand{\hmwkClass}{CS510 Languages and Low Level Programming}
\newcommand{\hmwkClassTime}{Spring 2016}
\newcommand{\hmwkClassInstructor}{Mark P. Jones}
\newcommand{\hmwkAuthorName}{Konstantin Macarenco}


\title{
    \vspace{2in}
    \textmd{\textbf{\hmwkClass:\ \hmwkTitle}}\\
        \normalsize\vspace{0.1in}\small{Due\ on\ \hmwkDueDate\ at 11:59pm}\\
        \vspace{0.1in}\large{\textit{\hmwkClassInstructor\ \hmwkClassTime}}
    \vspace{3in}
}

\author{\textbf{\hmwkAuthorName}}
\date{}

\renewcommand{\part}[1]{\textbf{\large Part \Alph{partCounter}}\stepcounter{partCounter}\\}

\begin{document}

\maketitle

\pagebreak

%\begin{enumerate}[(a), leftmargin = 0.7cm, nosep]
        \section{Topic 10.  Use practical case studies to evaluate and compare language design proposals.}

%STARTSTART

        With the lack of specific language for Low Level Programming, I picked two languages with
        Parallel Programming in mind (General C + MPI library, and Chapel - new domain specific
        language for parallel programming created by Cray). 
        Parallel programming complexity is similar, if not greater than
        LLP, with many potential issues on the top of regular mistakes there are Concurrency Issues
        such as race conditions, and deadlocks.
        MPI - message passing interface is a well known standard for external parallel computing.
        Chapel - is build upon C/MPI is much simpler - it is a modern high level language that hides
        all the intricacies of Message Passing. Chapel syntax is similar to Python, with many alike
        features. Chapel uses C/MPI as an intermediate layer, i.e. first it compiles to C.\\
    
        I Compare implementation of Jacobi-Laplace algorithm in C/MPI vs Chapel. 
        
        Jacobi-Laplace is a
        simple approach for solving Laplace equation with $O(n^3)$ complexity, used in many
        scientific applications. As $n\rightarrow \infty$ performance is greatly reduced, hence the
        desire to run it in parallel mode.\\
        Laplace equation : 
        ${\phi^{t+1}}_{i,j} = \cfrac{1}{4} ({\phi^{t}}_{i+1,j} + {\phi^{t}}_{i-1,j} +
        {\phi^{t}}_{i,j+1} + {\phi^{t}}_{i,j-1}), 0 < i, j< N$ , i.e. current cell in a matrix is
        equal to a quarter of the sum of it's neighboring cells.\\
    
        External im
        

        \begin{enumerate}[1.]
            \item Simple C/MPI implementation:
                \begin{itemize}
                    \item Complexity - implementing it in C/MPI requires matrix partition, 
                    \item Size - $\approx 620\ lines$ with comments
                \end{itemize}

            \item Simple Chapel implementation:
                \begin{itemize}
                    \item Complexity - 
                    \item Size - $\approx 60\ lines$ with comments
                \end{itemize}
        \end{enumerate}


%ENDEND
\end{document}
