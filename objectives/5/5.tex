\documentclass{article}

\usepackage{fancyhdr}
\usepackage{extramarks}
\usepackage{amsmath}
\usepackage{amsthm}
\usepackage{amsfonts}
\usepackage{tikz}
\usepackage[plain]{algorithm}
\usepackage{algpseudocode}
\usepackage{enumerate}

\usepackage{listings}
\usepackage{xcolor}
\usepackage{forest}
\usepackage[shortlabels]{enumitem}
     \setlist[enumerate, 1]{1\textsuperscript{o}}
\lstset { %
    language=C++,
    backgroundcolor=\color{black!5}, % set backgroundcolor
    basicstyle=\footnotesize,% basic font setting
}

%\usetikzlibrary{automata,positioning}
\usetikzlibrary{positioning,shapes,shadows,arrows,automata}

%
% Basic Document Settings
%

\topmargin=-0.45in
\evensidemargin=0in
\oddsidemargin=0in
\textwidth=6.5in
\textheight=9.0in
\headsep=0.25in

\linespread{1.1}

\pagestyle{fancy}
\lhead{\hmwkAuthorName}
\rhead{ (\hmwkClassInstructor\ \hmwkClassTime): \hmwkTitle}
\lfoot{\lastxmark}
\cfoot{\thepage}

\renewcommand\headrulewidth{0.4pt}
\renewcommand\footrulewidth{0.4pt}

\setlength\parindent{0pt}

%
% Create Problem Sections
%

\newcommand{\enterProblemHeader}[1]{
    \nobreak\extramarks{}{Problem \arabic{#1} continued on next page\ldots}\nobreak{}
    \nobreak\extramarks{Problem \arabic{#1} (continued)}{Problem \arabic{#1} continued on next page\ldots}\nobreak{}
}

\newcommand{\exitProblemHeader}[1]{
    \nobreak\extramarks{Problem \arabic{#1} (continued)}{Problem \arabic{#1} continued on next page\ldots}\nobreak{}
    \stepcounter{#1}
    \nobreak\extramarks{Problem \arabic{#1}}{}\nobreak{}
}

\setcounter{secnumdepth}{0}
\newcounter{partCounter}


\newcommand{\hmwkTitle}{Portfolio submission\ \#3, Topic 5}
\newcommand{\hmwkDueDate}{April 30, 2016}
\newcommand{\hmwkClass}{CS510 Languages and Low Level Programming}
\newcommand{\hmwkClassTime}{Spring 2016}
\newcommand{\hmwkClassInstructor}{Mark P. Jones}
\newcommand{\hmwkAuthorName}{Konstantin Macarenco}


\title{
    \vspace{2in}
    \textmd{\textbf{\hmwkClass:\ \hmwkTitle}}\\
    \normalsize\vspace{0.1in}\small{Due\ on\ \hmwkDueDate\ at 11:59pm}\\
    \vspace{0.1in}\large{\textit{\hmwkClassInstructor\ \hmwkClassTime}}
    \vspace{3in}
}

\author{\textbf{\hmwkAuthorName}}
\date{}

\renewcommand{\part}[1]{\textbf{\large Part \Alph{partCounter}}\stepcounter{partCounter}\\}

%
% Various Helper Commands
%

% Useful for algorithms
\newcommand{\alg}[1]{\textsc{\bfseries \footnotesize #1}}

% For derivatives
\newcommand{\deriv}[1]{\frac{\mathrm{d}}{\mathrm{d}x} (#1)}

% For partial derivatives
\newcommand{\pderiv}[2]{\frac{\partial}{\partial #1} (#2)}

% Integral dx
\newcommand{\dx}{\mathrm{d}x}

% Alias for the Solution section header
\newcommand{\solution}{\textbf{\large Solution}}

% Probability commands: Expectation, Variance, Covariance, Bias
\newcommand{\E}{\mathrm{E}}
\newcommand{\Var}{\mathrm{Var}}
\newcommand{\Cov}{\mathrm{Cov}}
\newcommand{\Bias}{\mathrm{Bias}}

\begin{document}

\maketitle

\pagebreak

%\begin{enumerate}[(a), leftmargin = 0.7cm, nosep]
        \textbf{Attempted Topic: }\\
        Describe the key steps in a typical boot process, including the role of a bootloader.\\

        Boot process described in class
        CPU INIT $\rightarrow$ BIOS/UEFI (or some other ROM BOOTLOADER) $\rightarrow$ SCAN 
        DEVICES $\rightarrow$ FIND MBR $\rightarrow$ LOAD MBR
        $\rightarrow$ MBR CODE (BOOTLOADER) $\rightarrow$ GRUB $\rightarrow$ KERNEL + MODULES; 
        is very general and used by almost all computer architectures.\\

        The sequence begins with CPU initialization, followed by self test and then \%eip  is 
        set to some predefined by the vendor address.
        Reset vectors used by various vendors:

        \begin{enumerate}[1.]
            \item The reset vector for the 8086 processor is at physical address FFFF0h (16 bytes 
                below 1 MB).
            \item The reset vector for the 80386 and later x86 processors is physical address
                FFFFFFF0h (16 bytes below 4 GB).
            \item The reset vector for PowerPC/Power Architecture processors is at an effective
                address of 0x00000100 for 32-bit processors and 0x0000000000000100 for 64-bit
                processors.
            \item The reset vector for SPARC version 8 processors is at an address of 0x00; the 
                reset vector for SPARC version 9 processors is at an address of 0x20 for power-on 
                reset, 0x40 for watchdog reset, 0x60 for externally initiated reset, and 0x80 for
                software-initiated reset.
        \end{enumerate}


        Then CPU Executes ROM bootloader (BIOS/EFI), which scans available devices for a Master Boot
        Record (MBR). Once record is found BIOS copies MBR to RAM and let MBR code to take over
        further execution.

        MBR Code - loads a bootloader from some known location on the disk. Bootloader is used 
        to prepare the computer for use. There are various types of bootloader used by 
        different vendors.

        \begin{enumerate}[1.]
            \item Microsoft NTLDR
            \item Apple BootX
            \item Linux GRUB
        \end{enumerate}

        All implementations have similar task, with, sometimes significant differences in functionality.
        An OS kernel expects certain information available for further boot like registers set
        to predefined values, available
        memory information, boot flags ans system permissions i.e. all system information
        required for successful work. For example 
        Linux expects boot info in multiboot standard. Multiboot allows to treat a kernel as 
        any other executable file in ELF
        standard. GRUB loads OS kernel according to the addresses defined in kernel's ELF
        linker script.
        \\
        Instead of listing field that multboot provides I extended bootinfo example to print all 
        information that available to kernel.

        After bootloader is it sets EIP to the address of the kernel and OS takes over. \\
        At this stage bootloader code is no longer needed and can be deleted/overwritten.

        Bootloader abstracts a Particular OS boot details from Vendor specific hardware,
        making kernel more portable.


        %\begin{enumerate}[1.]
        %\item The CPU is initialized
        %    \begin{enumerate}[1.]
        %        \item Initialize itself when power is first applied or when the system is reset
        %        \item Basic self test
        %        \item Initialize registers to known states
        %        \item including the instruction pointer/program counter (On IA32, for example,
        %eip is set to $0xFFFFFFF0$, so the computer can begin executing programs. And those
        %programs can initialize the devices. But only if those programs are in memory.
        %    \end{enumerate}
        %\item The memory contains the apps and data that we need
        %\item The devices are initialized and operational
        %\end{enumerate}
        %CCCCOOO1

    \end{document}
